\documentclass[12pt, reqno]{article}

\usepackage{amsmath,amsfonts,amssymb,amsthm}
\usepackage[margin=0.75in]{geometry}
\usepackage{hyperref}
\usepackage{fancyhdr}
\usepackage{apacite}
\usepackage{natbib}
\usepackage{times}
\bibliographystyle{apacite}
\urlstyle{same}

\renewcommand{\rmdefault}{ptm}

\begin{document}

%%%%%%%%%%%%%%%%%%%%%%%%%%%%%%%%%%%%%%%%%%%%%%%%%%%%%%%%%%%%%%%%%%%%%%%%%%%%%%%%%%%%%%%%%%%%%%%%%%%%

\pagestyle{fancy}

\lhead{2D Pendulum Equations of Motion}
\chead{}
\rhead{Egor Larionov}
\lfoot{}
\cfoot{\thepage}
\rfoot{}
Consider a 2D one-link pendulum system with no friction, attached to a ceiling and subject to the gravitational force. The Newton equations of motion balance the net force on the pendulum with the change in its momentum:
\begin{align*}
F + F_g = m \ddot{p},
\end{align*}
where $F$ is the constraint force of the hinge on the ceiling and $F_g$ is the force of gravity. As usual, $m$ is the mass of the pendulum and $p$ is the 2D position of its centre of mass. We can rewrite these as:
\begin{align}
m \ddot{p} - F = mg, \label{eq:1}
\end{align}
where $F_g = mg$, and $g$ is the gravitational acceleration constant. Here, the right-hand-side corresponds to known external forces applied to the system, while the left-hand-side collects the unknowns.

Let $\theta$ represent the tilt of the pendulum where $\theta = 0$ corresponds to the stable equilibrium configuration. Then let $\omega = \dot{\theta}$ the the angular velocity of the pendulum about its centre of mass. We can now write the angular equations of motions relating the relevant rotational quantities as follows:
\begin{align*}
\tau &= \frac{DL}{Dt} = \dot{L} \\
r_xF_{y} - r_yF_{x} &= I\dot{\omega},
\end{align*}
where $I$ is the moment of inertia making up angular momentum $L = I\omega$. In 2D, the torque, $\tau$, generated by the constraint is expressed in terms of the constraint force as $r_xF_{y} - r_yF_{x}$, where $r = (r_x, r_y)$ is the vector from the centre of the pendulum to the point of contact at the hinge, and $F = (F_x, F_y)$. As before, collecting unknowns on the left-hand-side gives us 
\begin{align*}
I\dot{\omega} - r_xF_{y} + r_yF_{x} = 0. \tag{*} \label{eq:2temp}
\end{align*}
For brevity, we denote $\tilde{r} = (r_y, -r_x)$, simplifying \eqref{eq:2temp} to
\begin{align}
I\dot{\omega} + \tilde{r} \cdot F = 0. \label{eq:2}
\end{align}


Finally, the hinge constraint ensures that the acceleration of the pendulum at the hinge, $\ddot{p_r}$, is zero:
\begin{align*}
0 &= \ddot{p_r} = \ddot{p} -  \dot{\omega}\tilde{r} - \omega^2r.
\end{align*}
Rearranging terms as before gives us:
\begin{align}
- \ddot{p} + \dot{\omega}\tilde{r} = -\omega^2r. \label{eq:3} 
\end{align}

We model the pendulum with a rectangular block, which has inertia $I = \frac{m}{12}(w^2 + h^2)$ about its centre of mass, where $w$ and $h$ are its width and height respectively.

Putting equations \eqref{eq:1}, \eqref{eq:2} and \eqref{eq:3} together, we get the following system:
\begin{align*}
\begin{pmatrix}
m\mathbf{I} &  0 & -\mathbf{I} \\
0  & I & \tilde{r} \\
-\mathbf{I} & \tilde{r}^{\top} & 0 \\
\end{pmatrix}
\begin{pmatrix}
\ddot{p}^{\top} \\ \dot{\omega} \\ F^{\top}
\end{pmatrix}
 =
\begin{pmatrix}
mg^{\top} \\ 0 \\ -\omega^2r^{\top}
\end{pmatrix},
\end{align*}
where $\mathbf{I}$ is the 2x2 identity matrix, and the superscript ${}^{\top}$ converts a vector into a column vector. Solving for $(\ddot{p}, \dot{\omega}, F)$ gives us the linear and angular accelerations, which we can use to update the pendulum configuration by one time step.




\end{document}
